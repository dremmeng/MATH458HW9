\documentclass{article}

\usepackage[margin=1in]{geometry}
\usepackage{fancyhdr, lastpage}
\usepackage{tikz}
\usepackage{amsmath,amssymb,amsthm}
\usetikzlibrary{calc}
\usepackage{enumitem}
\usepackage{soul}
\usepackage{multicol}

% Universes
\newcommand{\NN}{\mathbb{N}}
\newcommand{\ZZ}{\mathbb{Z}}
\newcommand{\QQ}{\mathbb{Q}}
\newcommand{\RR}{\mathbb{R}}
\newcommand{\CC}{\mathbb{C}}

% Groups commands
\newcommand{\inv}{^{-1}}
\newcommand{\lcm}{\mathrm{lcm}}
\newcommand{\lr}[1]{\langle #1 \rangle}
\newcommand{\Inn}{\mathrm{Inn}}
\newcommand{\iso}{\cong}
\newcommand{\normal}{\triangleleft}


%%%%%%%%%%%%%%%%%%%%%%%%%%%%%%%%%%%%%%%%%%%%%%%%%%%%%%%%%%%%%%
\setlength{\parindent}{0cm}
\pagestyle{fancy}
\lhead{MATH458 Abstract Algebra}
\rhead{Homework 9}

%%%%%%%%%%%%%%%%%%%%%%%%%%%%%%%%%%%%%%%%%%%%%%%%%%%%%%%%%%%%%%
\begin{document}
\section*{Homework 9}

Unless otherwise indicated, you must justify all answers/steps. See the Canvas assignment for more information about the homework requirements. 

\begin{enumerate}
    
    \item A ring element $a$ is called \emph{idempotent} if $a^2=a$. The following are problems regarding idempotent elements. Each problem is otherwise unrelated.
    \begin{enumerate}
        \item Prove that is $a$ is an idempotent ring element, then $a^n=a$ for all positive integers $n$. 
        
        \item Show that any idempotent element in a commutative ring with unity other than 0 or 1 is a zero-divisor. 

    \end{enumerate}

    \item Find all units, zero-divisors, idempotents, and nilpotent elements in ring $\ZZ_3\oplus \ZZ_6$. (Recall: an element $a$ is nilpotent if $a^n=0$ for some positive integer $n$.)

    
    \item Suppose that $a$ and $b$ belong to an integral domain $R$. If $a^m=b^m$ and $a^n=b^n$, where $m$ and $n$ positive integers that are relatively prime, prove that $a=b$. Note that the elements $a,b$ are not necessarily units, so we cannot assume $a\inv$ or $b\inv$  (or powers of $a\inv$ or $b\inv$) exist.
    
    \item The following are problems regarding the characteristic of a ring. 
    \begin{enumerate}
        \item Let $R$ be a ring with $m$ elements. Show that the characteristic of $R$ divides $m$. 
         
        \item Show that any finite field has order $p^n$, where $p$ is a prime. (Hint: Use facts about finite abelian groups.)
        
    \end{enumerate}

\end{enumerate}

\end{document}