% LaTeX Article Template - customizing page format
%
% LaTeX document uses 10-point fonts by default.  To use
% 11-point or 12-point fonts, use \documentclass[11pt]{article}
% or \documentclass[12pt]{article}.
\documentclass{article}

% Set left margin - The default is 1 inch, so the following 
% command sets a 1.25-inch left margin.
\setlength{\oddsidemargin}{0.25in}

% Set width of the text - What is left will be the right margin.
% In this case, right margin is 8.5in - 1.25in - 6in = 1.25in.
\setlength{\textwidth}{6in}

% Set top margin - The default is 1 inch, so the following 
% command sets a 0.75-inch top margin.
\setlength{\topmargin}{-0.25in}

% Set height of the text - What is left will be the bottom margin.
% In this case, bottom margin is 11in - 0.75in - 9.5in = 0.75in
\setlength{\textheight}{8in}
\usepackage{fancyhdr, lastpage}
\usepackage{tikz}
\usepackage{amsmath,amssymb,amsthm}
\usetikzlibrary{calc}
\usepackage{enumitem}
\usepackage{soul}
\usetikzlibrary{positioning}
\graphicspath{ {./} }
\setlength{\parskip}{5pt} 
\pagestyle{fancyplain}

% Universes
\newcommand{\NN}{\mathbb{N}}
\newcommand{\ZZ}{\mathbb{Z}}
\newcommand{\QQ}{\mathbb{Q}}
\newcommand{\RR}{\mathbb{R}}
\newcommand{\CC}{\mathbb{C}}

% Groups commands
\newcommand{\inv}{^{-1}}
\newcommand{\lcm}{\mathrm{lcm}}
\newcommand{\lr}[1]{\langle #1 \rangle}
\newcommand{\Inn}{\mathrm{Inn}}
\newcommand{\iso}{\cong}
\newcommand{\normal}{\triangleleft}
% Set the beginning of a LaTeX document
\begin{document}

\lhead{Drew Remmenga MATH 458}
\rhead{HW \#7}
%\lhead{Independent Study}
%\rhead{R Lab}


\begin{enumerate}

 \item A ring element $a$ is called \emph{idempotent} if $a^2=a$. The following are problems regarding idempotent elements. Each problem is otherwise unrelated.
    \begin{enumerate}
        \item Prove that is $a$ is an idempotent ring element, then $a^n=a$ for all positive integers $n$. $a$ is indempotent so $a^{2}=a$.Now for $m=1$. $a^{m}=a^{1}=a$. Now for $m=k+1$. $a^{k+1}=a^{k}a=aa=a$ since $a^{k}=a$ and $a^{2}=a$ by the induction hypothesis. So we can say it is true for all natural numbers. 
        
        \item Show that any idempotent element in a commutative ring with unity other than 0 or 1 is a zero-divisor. $a\neq 0,1$ Now $a(1-a)=a(1+(-a))=a+a(-a)=a-a^{2}=a-a$ since the ring is commutive with unity. Now we have $a-a=0$ So with $a\neq 0,1$ we have an element $(1-a)$ such that $a(1-a)=0$. So $a$ is a zero divisor. 

    \end{enumerate}

    \item Find all units, zero-divisors, idempotents, and nilpotent elements in ring $\ZZ_3\oplus \ZZ_6$. (Recall: an element $a$ is nilpotent if $a^n=0$ for some positive integer $n$.) Zero element of $\ZZ_3\oplus \ZZ_6$ is (0,0). Identity element of $\ZZ_3\oplus \ZZ_6$ is (1,1). Unit let $(a,b) and (c,d) \in \ZZ_3\oplus \ZZ_6$ such that $(a,b) \dot (c,d) = (1,1)$. Then we have $(ac,bd)=(1,1), ac=1, bd=1$. So $a$ is a unit and $b$ is a unit. Now we know $a$ is a unit in $\ZZ_n$ for some integer $n$ if $gcd(n,a)=1$ So 1,2 are units in $\ZZ_3$ and 1,5 are units in $\ZZ_6$. Hence the units of $\ZZ_3\oplus \ZZ_6$ are (1,1),(1,5),(2,1),and(2,5). Zero divisors: let (a,b) bye a zero divisor of $\ZZ_3\oplus \ZZ_6$. Then there exists non zero element (c,d) in $\ZZ_3\oplus \ZZ_6$ such that (ac,bd)=(0,0). So ac=0 and bd=0. 0 is the only zero divisor of $\ZZ_3$ as it is a field. the zero divisors of $\ZZ_6$ are 0,2,3,4. So our zero divisors of $\ZZ_3\oplus \ZZ_6$ are (0,0),(0,2),(0,3),(0,4). Indepotent: Let (a,b) be indempotent of $\ZZ_3\oplus \ZZ_6$. Then $(a,b)^{2}=(a,b)$ so $a^{2}=a$ and $b^{2}=b$. So $a$ and $b$ are indempotent in $\ZZ_3$ and $\ZZ_6$ respectfully. 0,1 are indempotent in $\ZZ_3$ 0,1,3,4 are indempotent in $\ZZ_6$. So the indempotents of $\ZZ_3\oplus \ZZ_6$ are (0,0),(0,1),(0,3),(0,4),(1,0),(1,1),(1,3), and (1,4). Nilpotent. Similarly if we can show (a,b) is nilpotent in $\ZZ_3\oplus \ZZ_6$ then a,b are nilpotent in $\ZZ_3$ and $\ZZ_6$. 0 is the only nilpotent element in $\ZZ_3$. 0 is the only nilpotent element in $\ZZ_6$. So (0,0) is the only nilpotent element of $\ZZ_3\oplus \ZZ_6$. 

    
    \item Suppose that $a$ and $b$ belong to an integral domain $R$. If $a^m=b^m$ and $a^n=b^n$, where $m$ and $n$ positive integers that are relatively prime, prove that $a=b$. Note that the elements $a,b$ are not necessarily units, so we cannot assume $a\inv$ or $b\inv$  (or powers of $a\inv$ or $b\inv$) exist. Because gcd(m,n)=1 $\exists x,y \in \ZZ$ such that mx+ny=1. $a^{1}=a^{mx+ny}=a^{mx}a^{ny}=(a^{m})^{x}(a^{n})^{y}=(b^{m})^{x}(b^{n})^{y}=b^{mx+ny}=b$ so $a=b$.
    
    \item The following are problems regarding the characteristic of a ring. 
    \begin{enumerate}
        \item Let $R$ be a ring with $m$ elements. Show that the characteristic of $R$ divides $m$. Let $r$ be the characteristic of $R$. Since the characteristic is $r$ there $\exists x \in R$ such that the group generated by $x$ under addition of the ring has size $r$. By lagrange theroem $r| \|R\|$ hence $r|m$. 
         
        \item Show that any finite field has order $p^n$, where $p$ is a prime. (Hint: Use facts about finite abelian groups.) Let $F$ be a finit efield. Le tthe smallest mutltipe of 1 that gives be $p$. $p$ is the characteristic of the field. We have $p.1=0$ Let it be possible p is not prime. then p=rs for some integers r and s less than p. p.1=(r.1)(s.1). Since p is the smallest possible integer such that p.1 =0 we have $(r.1)\neq0$ and $(s.1)\neq0$ but $(r.1)(s.1)=0$ this contradicts that $F$ is a field. Hence p is prime. Finally if q is prime other than p such that $q|\|F\|$ the since (R,+) is an abelian group $\exists x \in F$ such that $x\neq0, x+x...(q times)+x = 0$ hence x(q.1)=0. Since p is the characyeristic of the field and p doesn't divide q we have $q.1\neq0$ hence a contracdiction. Hence p is the only prime divisor of $\|F\|$.
        
    \end{enumerate}
\end{enumerate}
\end{document}